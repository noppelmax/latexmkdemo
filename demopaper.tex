
\documentclass{scrartcl}
\synctex=1

\usepackage{booktabs}
\usepackage{multirow}
\usepackage{csvsimple}

\usepackage{courier}
\usepackage{listings,color}
\definecolor{gray}{gray}{0.8}
\lstset{language=Perl}
\lstset{commentstyle=\textit}
\lstset{frame=shadowbox, rulesepcolor=\color{gray}}
\lstset{basicstyle=\footnotesize\ttfamily,breaklines=true}
\title{How to combine LibreOffice Calc with LaTex tables?}
\author{Maximilian Noppel}

\begin{document}


\maketitle
\bibliographystyle{plain}

\section*{What do we want?}

\section*{How do we make it work?}
Use the following perl script as \texttt{.latexmkrc} in the main folder of your paper. 
This file sets certain parameters for \texttt{latexmk}.
Also it alter the way \texttt{latexmk} handles \texttt{csv} slightly.
Below the code we discuss details of it.

\begin{lstlisting}[numbers=left]
  @default_files = ('demopaper.tex');
  $pdflatex      = 'pdflatex -synctex=1';
  $interaction   = "nonstopmode";
  
  add_cus_dep("ods","csv",0,"ods2csv");
  
  sub ods2csv {
      if($_[0] eq ""){
          return;
      }
      my $sourceods = "$_[0].ods";
      my $sourcecsv = "$_[0].csv";
      print "Processing $sourceods...\n";
      system("libreoffice --headless --convert-to csv \"". 
          $sourceods."\" --outdir tables");
      system("python src/preprocess_csv.py \"".$sourcecsv. 
          "\"  --inplace");
      return;
  }
  
  # Make sure all the dependencies get build initially
  # Starting from here our real customization starts.
  # First we scrap for all tables/*.ods files and put them in a list.
  my @files = <tables/*.ods>;
  my @dep_files = ();
  foreach (@files) {
    @dep_files = (@dep_files, $_)
  }
  
  # Then we iterate over the list
  # and check if every corresponding csv file
  # exists. If not we create it with the above
  # rules. Luckily latexmk takes care for 
  # rerunning the upper routine whenever
  # the .ods file changes, once the file exists...
  foreach (@dep_files) {
      my ($filetype) = $_ =~ /^.*\.(.*)$/igs;
      my ($filename) = $_ =~ /^(.*)\..+/igs;
  
      if($filetype eq "ods"){
          if(-e $filename.".csv"){
          }else{
              ods2csv($filename);
          }
      }
  }
\end{lstlisting}

\section*{Results}
Here you can see a table we created with our setup.

\begin{table}[!h]
  \caption{
    Our demotable from LibreOffice Calc
  }
  \label{tab:demotable}
  \centering
  \newcommand{\mymidruleA}{\cmidrule{1-5}}
\newcommand{\mymidruleB}{\cmidrule{2-5}}
\newcommand{\mymidruleC}{\cmidrule{2-2}}

\makeatletter
\NewDocumentCommand{\markifnotblank}{O{--} O{X} m m}{%
  \protected@edef\@tempa{#3}%
  \protected@edef\@tempb{#4}%
  \expandafter\notblank\expandafter{\@tempa}{#2}{%
    \expandafter\notblank\expandafter{\@tempb}{\color{gray!40}#2}{#1}%
  }
}

\newcommand{\tosuffix}[1]{%
  \protected@edef\@tempa{#1}%
  \expandafter\ifstrequal\expandafter{\@tempa}{1}{~(a)}{}%
  \expandafter\ifstrequal\expandafter{\@tempa}{2}{~(b)}{}%
  \expandafter\ifstrequal\expandafter{\@tempa}{3}{~(c)}{}%
}
\makeatother

\newcommand{\mycsvreader}[3][mymidruleB]{%
  \csvreader[
    head to column names,
    head to column names prefix = COL,
    filter = \not\equal{\csuse{COL#3}}{} \and \not\equal{\COLshowinpaper}{},
    late after line=\\\csuse{#1},
    late after last line=\\
  ]{#2}{}%
  {
  &
  \cite{\COLpaper}\tosuffix{\COLsuffix} & % Reference
  \markifnotblank{\COLhouses}{} & % Houses
  \markifnotblank{\COLstreets}{} & % Streets
  \markifnotblank{\COLwalls}{} & % Walls
  \COLcomment
  }
}

\begin{tabular}{
    ll
    ccc % threat model
    l % comment
  }
  \toprule
  &
  \multirow{2}{*}{\textbf{Paper} } &
  \multicolumn{3}{c}{\textbf{Which part?}} &
  \multirow{2}{*}{\textbf{Comment} } \\
  \cmidrule{3-5}
                  &
                  &
  \textbf{Houses}  &
  \textbf{Streets}  &
  \textbf{Walls} &
  
  \\
  \mymidruleA
  \multirow{3}{*}{Hello}
  \mycsvreader[]{tables/demotable.csv}{hello}
  \mymidruleA
  \multirow{3}{*}{Goodbye}
  \mycsvreader[]{tables/demotable.csv}{goodbye}
  \bottomrule
\end{tabular}
\end{table}


\bibliography{bib/references}
\end{document}
